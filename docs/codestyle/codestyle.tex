\documentclass[10pt,a4paper,titlepage,dutch]{report}
\usepackage{ae}
\usepackage{aecompl}
\usepackage[T1]{fontenc}
\usepackage[latin1]{inputenc}
\usepackage{a4}
\usepackage{geometry}
\geometry{verbose,a4paper}
\pagestyle{headings}
\setcounter{tocdepth}{3}
\setlength\parskip{\medskipamount}
\setlength\parindent{0pt}
\usepackage{float}
\usepackage{graphicx}
\usepackage{setspace}


% ----------------------------------------------------------------
\vfuzz2pt % Don't report over-full v-boxes if over-edge is small
\hfuzz2pt % Don't report over-full h-boxes if over-edge is small
% THEOREMS -------------------------------------------------------
%\newtheorem{thm}{Theorem}[section]
%\newtheorem{cor}[thm]{Corollary}
%\newtheorem{lem}[thm]{Lemma}
%\newtheorem{prop}[thm]{Proposition}
%\theoremstyle{definition}
%\newtheorem{defn}[thm]{Definition}
%\theoremstyle{remark}
%\newtheorem{rem}[thm]{Remark}
%\numberwithin{equation}{section}
% MATH -----------------------------------------------------------
\newcommand{\norm}[1]{\left\Vert#1\right\Vert}
\newcommand{\abs}[1]{\left\vert#1\right\vert}
\newcommand{\set}[1]{\left\{#1\right\}}
\newcommand{\Real}{\mathbb R}
\newcommand{\eps}{\varepsilon}
\newcommand{\To}{\longrightarrow}
\newcommand{\BX}{\mathbf{B}(X)}
\newcommand{\A}{\mathcal{A}}
% ----------------------------------------------------------------

\begin{document}

\title{lass\\Library for Acoustics Simulation Software\\C++ Style Guide}
\author{Group Acoustics\\Department of Information Technology (INTEC)\\Ghent University}
%\address{Department of Information Technology}
%\email{acoust\@intec.ugent.be}

%\thanks{}%
%\subjclass{}%
%\keywords{}%

%\date{}%
%\dedicatory{}%
%\commby{}%
% ----------------------------------------------------------------
%\begin{abstract}
%
%\end{abstract}

\maketitle

\tableofcontents

% ----------------------------------------------------------------
\chapter{Introduction}

If someone wants to write an introduction over here, please, feel
free to do so.

% ------------------------------------------------------------------------------

\chapter{Common Wisdom}

\begin{itemize}
    \item Every time a rule is broken, this must be clearly documented.
    \item We code and comment in American English (Note: we didn't say
anything on jokes).
    \item Don't code what you don't use.
    \item Don't code what you can't test.
    \item All code is evil.
    \item Fight code entropy.
    \item 80\% of the time is spent in 20\% of the code.
    \item Premature optimization is the root of all evil.
    \item The coffee machine is at the other end of the hallway.
    \item Common wisdom, n: something that is widely known but usually
ignored.
\end{itemize}

% ------------------------------------------------------------------------------

\chapter{Comments}

\begin{itemize}

\item Clear code is preferable to well-commented messy code. Every
effort should be made to make code clear through good design,
simplicity, good naming and layout. Comments are definitely
required where code is not clear.

\item Avoid compulsory comments like author, date, class names, function
names, and modification lists as they get outdated quickly and are
usually unnecessary. If required, try to use source control or
other tools to insert this type of information.

\item Use \verb|//| for comments, this will allow you to comment out big
blocks of code without interference using with the \verb|/**/|
construct.  Your doxygen comments will use \verb|/**/| though.

\item The generation of documentation will be based on the comments
written in the code itself. The automatic documentation generation
tool Doxygen is used with the Javadoc convention of documentation.
Look at www.doxygen.org for more information.  All public
interface code must therefore be documented.  These code blocks will be
written above the implementations.  Test code will be
excluded from the \verb|user manuals|.

\verb|/** a useless function|\\
\verb| *  @throw an exception is randomly thrown.|\\
\verb| */|\\
\verb|void fooBar()|\\
\verb|{|\\
\verb|    // ...|\\
\verb|}|

\end{itemize}

% ------------------------------------------------------------------------------

\chapter{Naming Conventions}

\section{typographical questions}

\subsection{class, struct, union}

Class, structure and union names start with a capital letter and
every new word with a capital.

\verb|class FooBar;|\\
\verb|struct FooBar;|\\
\verb|union FooBar;|\\

\subsection{variables and functions}

Variable and function names start with a lowercase letter and
every new word with a captial.

\verb|int fooBar;|\\
\verb|int fooBar();|\\

Private member variables of classes should have an underscore as
suffix.  Public member variables of structures don't.

\verb|class Foo|\\
\verb|{|\\
\verb|    int privatePart_;|\\
\verb|};|

\verb|struct Bar|\\
\verb|{|\\
\verb|    int publicPart;|\\
\verb|};|\\

Parameters of functions start with \verb|i| for input values,
\verb|o| for output values and \verb|io| for bidirectional
parameters

\verb|void fooBar( const A& iAnInputValue, |\\
\verb|             B& oSomeOutput, |\\
\verb|             C& ioABitOfBoth );|\\

\subsection{namespaces}

Namespace names use lowercase letters only and the different words
are seperated by underscores.

\verb|namespace foo_bar;|\\

\subsection{files}

Filenames use lowercase letters only and the different words are
separated by underscores.

\verb|foo_bar.h|\\

\subsection{macros}

Macro names use capitals only, the different words are seperated
by underscores, and they start with \verb|LASS_|.

\verb|LASS_FOO_BAR|\\

\subsection{common sense}

Never use underscores in names except for namespaces and macros.

Never use leading underscores or double underscores in any kind of
name, as it is prohibited by the ISO standard \cite{Cpp98}  in
section 17.4.3.1.2, well not that strictly, but strictly enough
for not doing it at all.

\verb|_DON_T|\\
\verb|DON__T|\\
\verb|DO_THIS|\\

Don't use ALL-CAPS for abbreviations, and definitely not in
namespaces!

\verb|MyISOObject // bad|\\
\verb|MyIsoObject // good|\\
\verb|my_HTML_namespace // bad|\\
\verb|my_html_namespace // good|\\

\section{What's in a name?}

In general names of classes, variables, constants should be nouns,
names of functions should contain a verb (except for accessors,
see \ref{accessors}

The wider the scope of the variable, the more descriptive it
should be. A counter can be called \verb|i| if it is only used in
a small loop, but it is unacceptable as a widely used variable in
a long function.

Don't use hungarian notation (where a code describing the
variable's type is prefixed to the variable name) for variables.

More common sense on naming can be found in Ottinger
\cite{Ottinger}

\subsection{Get and Set Accessors}

In most other code bases, you'll find the usual
\verb|get|/\verb|set| pairs as following:

\verb|class Bar|\\
\verb|{|\\
\verb|public:|\\
\verb|    const Foo& getFoo() const { return foo_; }|\\
\verb|    void setFoo( const Foo& iFoo ) { foo_ = iFoo; }|\\
\verb|private:|\\
\verb|    Foo foo_;|\\
\verb|};|\\

However, we do not do that for simple accessors.  Instead, we
write two accessors with the same name (\verb|foo|), one return a
constant reference, the other a non-constant reference.  As
following (BTW, notice that we never write code in the interface
...):

\verb|class Bar|\\
\verb|{|\\
\verb|public:|\\
\verb|    const Foo& foo() const;|\\
\verb|    Foo& foo();|\\
\verb|private:|\\
\verb|    Foo foo_;|\\
\verb|};|\\
\verb||\\
\verb|const Foo& Bar::foo() const|\\
\verb|{|\\
\verb|    return foo_;|\\
\verb|}|;\\
\verb||\\
\verb|Foo& Bar::foo()|\\
\verb|{|\\
\verb|    return foo_;|\\
\verb|}|\\

Now you can ask us: ''Why the heck do you still use accessor
functions then?  That's the same thing as putting all data in the
public interface.  You should use a POD structure instead!''.
Then, Tom will give you a calm answer (or he could just write it
down here ;-).

From the developer's point of view it is a bad idea to make public
variables.  Using accessors gives the opportunity to set
breakpoints whenever this may necessary or to execute some extra
code in debug mode.  From the user's point of view it allows for
transparency of the internal interface.  The internal storage
could be omitted and replaced by a computation.  Although this
goes against the following paragraph... it is preferred above
forcing the user to change his code after upgrading the library.

Anyway, that's for \emph{simple} accessors.  That are accessors that
are not time-expensive and don't have any side-effects on
\verb|Bar| except for altering the value \verb|foo_|.  All
accessors that are expensive, or have side-effects will still use
the \verb|get|/\verb|set| pair (However, you can also mix, see
below).

\verb|class Bar|\\
\verb|{|\\
\verb|public:|\\
\verb|    const Foo& getFoo() const;|\\
\verb|    void setFoo( const Foo& iFoo() );|\\
\verb|private:|\\
\verb|    Foo foo_;|\\
\verb|};|\\
\verb||\\
\verb|const Foo& Bar::getFoo() const|\\
\verb|{|\\
\verb|    // doing some sloooooooooow stuff here !!!|\\
\verb|    return foo_;|\\
\verb|}|;\\
\verb||\\
\verb|void Bar::setFoo( const Foo& iFoo() )|\\
\verb|{|\\
\verb|    // doing some slooow stuff,|\\
\verb|    // or something that has impact on more than foo_ alone|\\
\verb|    foo_ = iFoo;|\\
\verb|}|\\

Anyway the idea is, we \emph{hate} set 'n get. So if some accessors
only come in the \verb|get|/\verb|set| variants, we should realize
-- as clients -- that something expensive or tricky is done here.

For the rare -- or not so rare -- cases getting something is cheap
but setting something is not (or the other way around), you may
also mix both schemes.

\verb|class Bar|\\
\verb|{|\\
\verb|public:|\\
\verb|    const Foo& foo() const; // cheap|\\
\verb|    void setFoo( const Foo& iFoo() ); // not so cheap|\\
\verb|private:|\\
\verb|    Foo foo_;|\\
\verb|};|\\

However, for the \emph{setters} there may only be one function, same story for
the \emph{getters}.  The same semantics may not be mixed into two
functions.  If the accessor is expensive it is expensive and not
dependent on the used member function.

Again set 'n get are not preferred.  Imagine a time series object
containing octave band values.  If you would like to change only
one value you could write (not using set 'n get):

\verb| TimeSeriesObject.value(iTime).band(iBand) = newBandValue|;

If you would use either get or set for the accessor above code
would be much longer and less obvious.

\verb| tempOctaveBand = TimeSeriesObject.getValue(iTime);|\\
\verb| tempOctaveBand.setBand(iBand) = newBandValue;|\\
\verb| TimeSeriesObject.setValue(iTime) = tempOctaveBand;|\\

Hmmm, not like we want it as a user.



% ------------------------------------------------------------------------------

\chapter{Files}

\section{file\_names}

We use the extension \verb|.h| for header files, \verb|.cpp| for
source files that are not included by other files (in other words:
translation units), and \verb|.inl| for source files that are
included by others (e.g.: source files of templated classes,
inlined functions, ...).

Every class should have a separate source file and header file
that matches the name of the class. We use small letters only, and
different words are separated by a single underscore. For example
\verb|class FooBar| should be defined in \verb|foo_bar.h| and be
located in \verb|foo_bar.cpp| (or other chosen extensions).

All definitions should reside in source files. The header files
should declare an interface, the source file should implement it.
When looking for an implementation, the programmer should always
know that it is found in the source file. Inline functions can be
defined in the \verb|.inl| file.

\section{Source Tree}

TO BE WRITTEN BY TOM?

A Source Tree is a Tree containing the Source ;o)

\section{Including Headers}

Use relative paths in \verb|#include| statements for all internal
headers of \verb|lass|.  Note: this is only for \verb|lass|
development!  Clients must include files of \verb|lass| as library
files.

\verb|#include "../lass_common.h" // good|\\
\verb|#include "../prim/vector3.h" // good|\\
\verb|#include <lass/prim/vector3.h> // bad|\\

Use new-style headers for \verb|STL| library files, so you get
your stuff in the \verb|std| namespace.

\verb|#include <iostream> // good|\\
\verb|#include <iostream.h> // bad|\\

Some compilers, like our Microsoft one, don't understand there's a
difference between \verb|#include <cmath>| and \verb|#include|
\verb|#<math.h>|.  \verb|cmath| should have everything of
\verb|math.h| included in \verb|std| namespace.  Except for
Microsoft, they have both included in the global namespace.
Therefore, use the old fashion headers only (\verb|math.h|) if
you're dealing with such kind of headers.

\verb|#include <math.h> // good|\\
\verb|#include <cmath> // bad|\\

Include all other external libraries as libraries.

\verb|#include <boost/scoped_ptr.hpp>|\\

Try to avoid putting \verb|#include| in header files if a forward
declaration is sufficient.  For example:

\begin{verbatim}
class bar;
class Foo
{
    void func1(const Bar& iBar);
};
\end{verbatim}

is preferable to

\begin{verbatim}
#include "bar.h"
class Foo
{
    void func1(const Bar& iBar);
};
\end{verbatim}

Include statements must be located at the top of a file only.
Avoid unwanted compilation side effects by "hidden" include
statements deep into a source file.

\section {Guardian of Inclusion}

Each file \verb|/num/foo_bar.h| that can be included by others
(\verb|.h| and \verb|.inl| files) must have an inclusion guard as
following. Notice how it states the subdirectory (subnamespace)
\verb|num|.

\verb|#ifndef LASS_GUARDIAN_OF_INCLUSION_NUM_FOO_BAR_H|\\
\verb|#define LASS_GUARDIAN_OF_INCLUSION_NUM_FOO_BAR_H|\\
\verb|...|\\
\verb|#endif|\\

Yeah yeah, you can also use \verb|#pragma once| (and if you're on
a microsoft compiler, you should! :-) but don't use it as an
alternative for the inclusion guard, since not all compilers
recognize this.

% ----------------------------------------------------------------------

\chapter{Clear Code Style}

Poorly written \verb|C++| can be very difficult to understand. It
can be simplified by not using many of the features of \verb|C++|:
some of which are leftover from \verb|C|, some which are poorly
understood by many developers and some which are unnecessary.

\verb|#ifdef/#endif| should not be used in a function to make it
portable. Use wrapper classes to cover platform-dependent code.

Macro's are evil!  So they are to be avoided. Always use constants
or functions if possible as they are clearer and are more easily
understood by software tools such as debuggers and source
browsers.

However, macro's still have their usage, and you should not
hesitate to use them if necessary, or can make the code more
clear.  An example are the debug macro's like \verb|LASS_ASSERT|,
\verb|LASS_THROW|, ...  Their functionality can't be mimicked by
functions.  Another example is
\verb|LASS_IO_BINARY_O_STREAM_INSERTOR| that prevents us of copy
pasting almost identical code over and over again.  Still, be
carefully when choosing their names.  For macro's that must be
used by clients short names are preferred, but for
"implementation" macro's that are only used for generating code
(like \verb|LASS_IO_BINARY_O_STREAM_INSERTOR|) the most ugly
possible names should be choosen (to prevent nameclashes).

Avoid \verb|C| library functions which are superceded by modern
objects, e.g. use \verb|std::string| class methods rather than
\verb|strcmp|, use STL streams instead of \verb|C| runtime
functions.

Don't use optimization hint keyword \verb|register|.

Use of keyword \verb|volatile| may be necessary to avoid problems
with optimizing compilers and multithreaded apps, but avoid if
possible.

Don't use run time type information (RTTI) unless you have to. The
overhead is unnecessary and such a mechanism shouldn't normally be
needed with good software design and template-based collections.

Avoid using \verb|goto|, however it can be useful to have a goto
to jump exit out of a series of nested loops. A preferable
solution would be to place the loops into a separate function with
a return statement to end the looping.

Use \verb|?:| instead of \verb|if| statements in case of
assignments and \verb|return| statements, on the condition you can
keep it simple.

\verb|a = (b == 1) ? c : d;|\\
\verb|return b == 1 ? c : d;|\\

\section{Variables}

Variables should not be reused for different purposes within the
same routine.

Variables should be declared close to where it is used. Don't put
all the variable declarations at the beginning of a function.

Never use global variables in projects larger than trivial. You
can't tell where or when they are used and they might have
initialization sequence problems. Try using static class members
and/or a singleton class to manage widely used variables.

Constants should be defined in the smallest scope possible, for
example if a constant is only used in one function it should be
defined in the function.

Use the \verb|C++| type \verb|bool| for boolean variables.

\section{Classes}

We all know that in \verb|C++| there isn't a difference between
\verb|class| and \verb|struct| except for the default access
rights.

\verb|class Foo|\\
\verb|{|\\
\verb|    int privateBar_;|\\
\verb|};|

is identical to

\verb|struct Foo|\\
\verb|{|\\
\verb|private:|\\
\verb|    int privateBar_;|\\
\verb|};|

However, we choose to distinguish between \verb|class| and
\verb|struct| for clarity.  \verb|class| must be used for all
structures that don't have public data members.  \verb|struct|
should be used for structures with public data like POD's (plain
old data structures).  Notice that a structure with public data
can still have structors and other methods.



% ----------------------------------------------------------------------

\chapter{Formatting and Layout}

\begin{itemize}

\item Layout should be written for maximum code readability.

\item Maximum number of characters per line?

\item Don't check equivalence backwards It may be safer but it is more
difficult to read.

\verb|if (false == variableA) // bad|\\
\verb|if (variableA == false) // good|

\item Each variable declaration needs a line of its own.

\verb|int i;|\\
\verb|int j;|

\item If the variable declaration is a pointer, place the star next to
the type, not the name.

\verb|char* p; // good|\\
\verb|char *q; // bad|\\
\verb|char * r; // bad|

\item Use tabs instead of spaces to indent a source file. The tab size
for source files should be set to every four spaces.

\item Place a space around both sides of binary operators such as
\verb|+| and \verb+||+.

\verb|c = 2 * a + b;|\\
\verb+if (x || y)+

\item unary operators should "stick" to the operand, but have an extra

\item Use parentheses wherever they may remove possible ambiguity in the
minds of the reader.

\item Matching braces ('\verb|{|' '\verb|}|') should line up vertically
inline with the parent statement.

\item Place single spaces immediately after commas in routine parameter
lists.

\verb|void func1( const std::string& iName, long iNumberOfFoobars )|\\
\verb|{|\\
\verb|    if (...)|\\
\verb|    {|\\
\verb|        long i;|\\
\verb|        ...|\\
\verb|    }|\\
\verb|}|

\item Avoid one-line \verb|if| statements.  So you can easily add more
lines of code to it, and above all: you can more easily set
breakpoints on it.

\verb|if (a == 5) b = 6; // bad|\\
\verb|if (a == 5)|\\
\verb|{|\\
\verb|    b = 6; // good|\\
\verb|}|

\end{itemize}


% ----------------------------------------------------------------

\chapter{Good Coding Practices}

\begin{itemize}

\item Prefix versus postfix operators. This is one of all times, yet
it's very simple: you should only use the postfix operator if you
need it to return the old value before increment or decrement, in
all other cases you probably should use the prefix operator.  The
best example is a \verb|for| loop.  You can almost always use the
prefix operator, so you should.  Why?  Because the prefix operator
can be more optimized than the postfix one.  "But ... but ... I
looked in the assembly code and \verb|i++| is as optimized as
\verb|++i| for integers", you say.  Sure, but you're not always be
using these operators on integers, so you'd better learn the right
habit right now.

\verb|for (int i = 0; i < size; ++i)|\\
\verb|{|\\
\verb|    sum += vector[i];|\\
\verb|}|

\item Safe \verb|C++| coding routines and their parameters should be
declared \verb|const| wherever possible.

\item Use the \verb|LASS_ASSERT| macro to test assumed or expected
conditions.

\item Assignment operators (and operators like \verb|+=|, HOW DO YOU
CALL THESE?) should return a reference to the object

\verb|Foo& Foo::operator=( const Foo& iOther )|\\
\verb|{|\\
\verb|    ...|\\
\verb|    return *this;|\\
\verb|}|

\item Classes that will be inherited from should have a virtual
destructor.

\item If a class defines either an assignment operator or a copy
constructor it probably should define both.

\item If a class needs a destructor, it often needs a copy constructor
and assigment operator as well.

\item Don't throw objects allocated from the heap. You don't know if
they will be freed.

\item Use accessors instead of public variables. If you have a function
instead of a public member you can add a breakpoint to the
function to see when it is being used.

\item Use references instead of pointers if possible. A reference cannot
be changed to point to another object so that you know it always
points to the same object. And above all, a reference can never be
NULL! This is especially useful for function parameters.

\item use \verb|std::string| instead of character arrays.

\end{itemize}

% ----------------------------------------------------------------

\chapter{On Compilers, IDE's and other stuff}

We use Microsoft Visual Studio .NET 2002 as IDE.  For compilers,
we support both the Microsoft Visual C++ .NET 2002 (a.k.a.~Visual
C++ 7.0 or VC7) and the Intel C++ Compiler 7.1 for Windows.
Furthermore, we also use STLport for the STL implementation, and
although we still try to avoid it, you might need the Boost
libraries too.

\section{Installation}

\subsection{Microsoft Visual Studio .NET 2002 Enterprise Developer}

\begin{itemize}

\item First, grab the four ISO's of Microsoft Visual Studio .NET
2002 Enterprise Developer (MSVS7 here after).  They're available
on arcserv. Or, alternative, if you're in the Acoust workgroup,
check out the warehouse on Gibson \cite{Gibson}. But of course,
you could ask one of the people around if they have it already on
CD :).

\item Find yourself the `component update disk' (or something)\dots You
can use the fifth disk of e.g. the professional edition of MSVS7,
or use the one on gibson (`the darned components update disk' in
the same folder as the ISO's). It's unpacked there, so if you're
in the neighborhood, you don't have to burn it on a CD.

\item As last, you will need the service pack.  Available on MSDN, or
again on gibson.

\item Alright!  Once you have all that, you're ready to install.  Insert
the first disk of MSVS7.  Probably, you'll get a nice `3 steps to
install' window.  Before the setup lets you install MSVC7EntDev
(which is step 2), you first have to update the darned components.
OK, no problem: click on that step and do it! Normally, it should
be enough if you browse to the contents of that disk on Gibson.

\item The next step (2) is installing MSVS7 itself.  It's four disks, so
take it easy: it's gonna take a while (hihi :-)  Oh yeah, a
typical setup will eat 2 GB of your hard disk.  Though, you can
safely skip some parts.  All you need is Visual C++, and some
extra stuff of course.

\item Once you have done that, you can try step three, or install the
service pack.  OK, that's all.  Fire it up!

\end{itemize}

\subsection{Intel C++ Compiler 7.1 for
Windows}

\subsection{STLport}

\begin{itemize}

\item Currently, we're using STLport 4.5.3 \cite{STLport}.  Grab the
STLport-4.5.3.zip package from anywhere you might find it
\cite{STLport, Lass}. Of course, like with all zips, you have to
unzip it to a nice folder now. We'll assume you unzip it to
\verb|C:\dev|. You'll get a folder \verb|C:\dev\STLport-4.5.3|.

\end{itemize}

Before we continue, we must fix some parts of STLport.

\begin{itemize}

\item If you try to use \verb|<typeinfo>| with MSVC7, you might find out
that it can't find \verb|type_info|, \verb|bad_cast| or
\verb|bad_typeid| in the namespace \verb|_STL|. To solve this,
apply next hack: Open \verb|C:\dev\STLport-4.5.3\typeinfo| and
write on the very first non-comment line: \verb|#include|
\verb|<typeinfo.h>|. (Note, STLPort said that 4.5.1 should work
alright but that 4.5.3 is not properly configured for MSVC7).

\item Currently, there's still a bug with the \verb|std::complex<>|
template.  It seems you can't check if two complex numbers are
inequal: \verb|a != b|.  To fix this, you should open
\verb|C:\dev\STLport-4.5.3\| \verb|src\_complex.h|, browse to line
736 and comment the \verb|#ifndef| guard around \verb|operator!=|:

\verb|//#ifdef _STLP_FUNCTION_TMPL_PARTIAL_ORDER|\\
\verb||\\
\verb|template <class _Tp> |\\
\verb|inline bool _STLP_CALL operator!=(const complex<_Tp>&|
\verb|__z1,|\\
\verb|                                  const complex<_Tp>& __z2) {|\\
\verb^  return __z1._M_re != __z2._M_re || __z1._M_im !=^
\verb|__z2._M_im;|\\
\verb|}|\\
\verb||\\
\verb|//#endif /* _STLP_FUNCTION_TMPL_PARTIAL_ORDER */|

\end{itemize}

Now it's time to build it (alternative, you can also grab the
builded libraries from `libs$\backslash$ext' on LASS's home on the
fileserver \cite{Lass}.  If you do, remember to copy the dlls to
your system folder):

\begin{itemize}

\item fire up a command prompt.

\item invoke \verb|vcvars32.bat|.  You can find it in
\verb|C:\Program| \verb|Files\Microsoft| \verb|Visual|
\verb|Studio| \verb|.NET\Vc7\bin|.

\item go to \verb|C:\dev\STLport-4.5.3\src| and execute the
following commands:

\verb|nmake -f vc7.mak clean install|\\
\verb|nmake -f intel50.mak clean install|\\

\end{itemize}

Once you've build STLport, you need to tell Visual Studio where to
find it.

\begin{itemize}

\item In the menu, go to \emph{Tools > Options > Projects > VC++
Directories}.

\item Select \emph{include files} in \emph{Show directories for:}
and add \verb|C:\dev\STLport-4.5.3\stlport| to the list. Make sure
it is listed before any of the Visual Studio include folders (the
ones starting with \verb|$(VCInstallDir)|).

\item Select \emph{library files} in \emph{Show directories for:}
and add \verb|C:\dev\STLport-4.5.3\lib| to the list. Again, make
sure it is listed before any of the Visual Studio include folders.

\end{itemize}

\subsection{C++ Boost}

\begin{itemize}

\item At this moment, we're also using the Boost libraries version
1.30.2 \cite{Boost} As with STLport, grab the boost-1.30.2.zip
from anywhere you might find it \cite{Boost, Lass}.  Again unzip
it \dots  If you unzip it to \verb|C:\dev|, you'll get a new
folder \verb|C:\dev\boost-1.30.2|.  Groovy!

\item Most of Boost is implemented in headers only, so you can already
use a lot of it by setting the correct include directories in
Visual Studio: Tools > Options > Projects > VC++ Directories >
Show directories for include files.  Add a new line on top and
browse it to \verb|C:\dev\boost-1.30.2|.  Now you can already use
e.g.~\verb|boost::scoped_ptr<>| by including
\verb|<boost/scoped_ptr.hpp>|.  For other stuff, Boost must be
build.  Take a deep breath and stick with me here \dots

\item First, make sure you've installed MSVS7, STLPort 4.5.3 and that
you've patched STLPort as described above.

\item Now it's time to patch Boost :)  open the following file:

\verb|C:\dev\boost-1.30.2\libs\regex\src\cpp_regex_traits.cpp|

browse to the line

\verb?#if defined(__SGI_STL_PORT) || defined(_STLPORT_VERSION)?

and change it to (one line):

\verb?#if (defined(__SGI_STL_PORT) || defined(_STLPORT_VERSION))?\\
\verb?&& !defined(BOOST_MSVC)?

Insert under this line (and above \verb|using namespace std;|) the
following:

\verb|using stlport::isspace;|\\

\item Now everything is patched, let's build Boost\dots  Very very short: Get the \verb|bjam.exe|, Invoke
\verb|vcvars32.bat|, go to boost directory and do the command (the
following three lines are one command!):

\verb|C:\dev\boost-1.30.2>bjam "-sTOOLS=msvc-stlport"|\\
\verb|"-sMSVC_ROOT=c:\Program Files\Microsoft Visual Studio .NET\VC7"|\\
\verb|"-sSTLPORT_PATH=c:\dev" "-sBUILD=<stlport-iostream>on"|\\

\end{itemize}

\section{Project Properties for lass}

In this section, we'll enumerate the different project settings
you need to build static libraries for \verb|lass| (warning: this
is all MSVC70 stuff). Let's go ... Click right on your project and
select `properties'.

\subsection{Configuration Names}

The following settings rely on a fixed scheme of configuration
names.  Instead of \verb|Debug| and \verb|Release| you have to use
for each configuration a name as following:

\emph{platform}\verb|_|\emph{compiler}\emph{[debug]}

With:
\begin{itemize}
  \item \emph{platform} indicates for what platform this
configuration will be used.  It must be equal to the value of the
macro \verb|LASS_PLATFORM|.  For example, if the configuration is
for the windows platform, \emph{platform} will be equal to
\verb|win32|.
  \item \emph{compiler} indicates on what compiler this
  configuration will be used.  It must be equal to the value of
  the macro \verb|LASS_COMPILER|.  For example, if you're using
  the Microsoft Visual C++ 7.0 compiler, this will be equal to
  \verb|vc7|
  \item \emph{debug} indicates wether this is a configuration
  for a debug or a release build.  In case of a debug build,
  \emph{debug} is equal to \verb|_d|, otherwise it's an empty
  string.
\end{itemize}

For a configuration for the windows platform, Intel C++ 7.0
compiler and a debug build, the configuration name would be
\verb|win32_intel700_d|.  For the release build this would reduce
to \verb|win32_intel700|.

Notice that each configuration must be used for a particular
combination of platform, compiler and build. You can't use the
same configuration for two different compilers.

Why is this configuration name so important?  This is because this
name is used to generate the name of the static lib that will be
build.  However, the same name must be generated by the macro
tools in the common headers of these libraries that automatically
link up to these libraries when used.  Because these names are
generated independently, we have to use a fixed naming scheme to
let them match.

\subsection{Common settings}

First, some common settings that are valid for all builds.

General

\begin{tabular}{|l|l|}
  \hline
  Output Directory & .$\backslash$\$(ConfigurationName)\\
  Intermediate Directory & .$\backslash$\$(ConfigurationName)\\
  Configuration Type & Static Library (.lib)\\
  \hline
\end{tabular}

C++

\begin{tabular}{|l|l|}
  \hline
  \emph{Language} & \\
  Default Char Unsigned & Yes\\
  Force Conformance in For Loop & Yes\\
  \hline
  \emph{Output Files} & \\
  ASM List Location & \$(OutDir)$\backslash$\\
  Object File Name & \$(OutDir)$\backslash$\\
  Program Database File Name & \$(OutDir)$\backslash$\$(TargetName).pdb\\
  \hline
\end{tabular}

Librarian

\begin{tabular}{|l|l|}
  \hline
  \emph{General} & \\
  Output File & \$(OutDir)/\$(SolutionName)\_\$(ProjectName)\_\$(ConfigurationName).lib \\
  \hline
\end{tabular}



\subsection{Microsoft versus Intel}

\subsubsection*{Microsoft VC7}
Build Events
\begin{tabular}{|l|l|}
  \hline
  \emph{Post-Build Event} & \\
  Command Line & copy ``\$(TargetPath)'' ``\$(SolutionDir)..$\backslash$lib$\backslash$\$(TargetFileName)''\\
  & copy ``\$(TargetDir)$\backslash$\$(TargetName).pdb'' ``\$(SolutionDir)..$\backslash$lib$\backslash$\$(TargetName).pdb''\\
  Description & Collecting library file \$(TargetFileName) ... \\
  \hline
\end{tabular}

\subsubsection*{Intel intel700}
Build Events
\begin{tabular}{|l|l|}
  \hline
  \emph{Post-Build Event} & \\
  Command Line & copy ``\$(TargetPath)'' ``\$(SolutionDir)..$\backslash$lib$\backslash$\$(TargetFileName)''\\
  Description & Collecting library file \$(TargetFileName) ... \\
  \hline
\end{tabular}



\subsection{Debug versus release builds}

\subsubsection*{Debug}

C++

\begin{tabular}{|l|l|}
  \hline
  \emph{General} & \\
  Debug Information Format & Program Database for Edit \&
  Continue\\
  \hline
  \emph{Code Generation} & \\
  Runtime Library & Multi-threaded Debug DLL\\
  \hline
\end{tabular}

\subsubsection*{Release}

C++

\begin{tabular}{|l|l|}
  \hline
  \emph{General} & \\
  Debug Information Format & Program Database\\
  \hline
  \emph{Code Generation} & \\
  Runtime Library & Multi-threaded DLL\\
  \hline
\end{tabular}

\subsection {Precompiled Headers}

Here we'll describe how to used precompiled headers correctly.
Say, we have a project called \verb|foo| we want to compile with
precompiled headers. Create two files \verb|foo_common.h| and
\verb|foo_common.cpp|. The former will be the `precompiled'
header, the latter will perform the precompilation.

Select your project and open its properties (make sure you
selected your project so you alter the settings for all files).
Apply the following settings, you can do this for all
configurations at once:

\begin{tabular}{|l|l|}
  \hline
  \emph{C++} & \\
  Precompiled Headers & Create/Use Precompiled Header: Use Precompiled Header\\
  & Create/Use PCH Through File: foo\_common.h\\
  & Precompiled Header file:
  \$(IntDir)$\backslash$\$(TargetName).pch\\
  \hline
\end{tabular}

Select \verb|foo_common.cpp| only and change its properties to:

\begin{tabular}{|l|l|}
  \hline
  \emph{C++} & \\
  Precompiled Headers & Create/Use Precompiled Header: Create Precompiled Header\\
  \hline
\end{tabular}

As the last step, you must write in every translation unit
(\verb|.cpp| files) the following include on the first
\emph{non-comment} line:

\verb|#include "foo_common.h"|

\subsection{miscellaneous}

MSVC7 sometimes doesn't understand you want to enable RTTI.  Like
anyone, you go to the project properties > C/C++ > Language and
you enable Run-Time Type Info.  Unfortunately, you might find out
that doesn't work.  You might get warnings like: ``warning C4541:
`typeid' used on polymorphic type `foo' with /GR-; unpredictable
behavior may result''.  Unpredictable behavior, that is: crashing.
You don't want that of course, so you go again in the properties
and you check the commandline.  You'll find out that it indeed
enables RTTI as requested, /GR is mentioned in the commandline.
But!  If you check the buildlog, you notice it suddenly
disappeared!  Where did it go?  Nobody knows.  Fortunately, the
solution is simple: add /GR explicitly to the command line in the
project properties (now it will be there twice, but the first one
disappears, remember?), and your problem is solved ...  (Addendum:
well not quite yet.  you might be doomed forever on this ...)



% ----------------------------------------------------------------
%\bibliographystyle{amsplain}
%\bibliography{}

\begin{thebibliography}{99}

\bibitem{Boost} C++ BOOST, http://www.boost.org.
\bibitem{Cpp98} ISO/IEC 14882 (1998), {\it Programming Languages --
C++}.
\bibitem{Gibson} GIBSON,
$\backslash$$\backslash$gibson$\backslash$warehouse.
\bibitem{Lass} LASS,
$\backslash$$\backslash$Fileserver-urb$\backslash$acoust$\backslash$software$\backslash$intern$\backslash$lass.
\bibitem{Ottinger} OTTINGER T. (1997), {\it Ottinger's Rules for Variable
and Class Naming}, Object Mentor, Inc.,
http://www.objectmentor.com/resources/articles/naming.htm.
\bibitem{STLport} STLPORT, http://www.stlport.com.

\end{thebibliography}


\end{document}
% ----------------------------------------------------------------
